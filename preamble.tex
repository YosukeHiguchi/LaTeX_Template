\documentclass[11pt, oneside]{jsarticle}
\usepackage[truedimen,margin=25mm]{geometry}
\usepackage[dvipdfmx]{graphicx, color}
\usepackage{bmpsize}
\usepackage{amssymb, amsmath}
\usepackage{multirow}
\usepackage{slashbox}
\usepackage{here}
\usepackage{listings, jlisting}
\lstset{
    language = ,
    backgroundcolor = , %背景色と透過度 {/color[gray]{.90}}
    breaklines = true,
    breakindent = 10pt,
    basicstyle = \ttfamily\scriptsize, %標準の書体
    commentstyle = {\itshape \color[cmyk]{1,0.4,1,0}}, %コメントの書体
    classoffset = 0, %関数名等の色の設定
    keywordstyle = {\bfseries \color[cmyk]{0,1,0,0}}, %キーワード(int, ifなど)の書体
    stringstyle = {\ttfamily \color[rgb]{0,0,1}}, %表示する文字の書体
    %枠 "t"は上に線を記載, "T"は上に二重線を記載
    %他オプション:leftline,topline,bottomline,lines,single,shadowbox
    frame = tbrl,
    framesep = 5pt, %frameまでの間隔(行番号とプログラムの間)
    numbers = left, %行番号の位置
    stepnumber = 1, %行番号の間隔
    numberstyle = \tiny, %行番号の書体
    tabsize = 4, %タブの大きさ
    captionpos = t
}

\setlength{\tabcolsep}{5mm}

\def\vector#1{\mbox{\boldmath $#1$}}
