\documentclass[11pt, oneside]{jsarticle}
\usepackage[truedimen,top=20truemm,bottom=20truemm,left=25truemm,right=25truemm]{geometry}
\geometry{a4paper}
\usepackage[dvipdfmx]{graphicx, color}
\usepackage{bmpsize}
\usepackage{amssymb, amsmath}
\usepackage{multirow, multicol}
\usepackage{slashbox}
\usepackage{here}
\usepackage{listings, jlisting}

%ページ書式(分数形式)
\usepackage{fancyhdr}
\usepackage{lastpage}
\fancypagestyle{mypagestyle}{
    \lhead{}%ヘッダ左を空に
    \rhead{}%ヘッダ右を空に
    \cfoot{\thepage/\pageref{LastPage}}%フッタ中央に"今のページ数/総ページ数"を設定
    \renewcommand{\headrulewidth}{0.0pt}%ヘッダの線を消す
}
\pagestyle{mypagestyle}
\pagenumbering{arabic}

%数式番号を節毎に分けてリセットする
\def\theequation{\thesection.\arabic{equation}}
\makeatletter
    \@addtoreset{equation}{section}
    %図番号を節毎に分けてリセットする
    \renewcommand{\thefigure}{\thesection.\arabic{figure}}
    \@addtoreset{figure}{section}
    %表番号を節毎に分けてリセットする
    \renewcommand{\thetable}{\thesection.\arabic{table}}
    \@addtoreset{table}{section}
\makeatother

%プログラム挿入の書式
\lstset{
 	language = ,
 	backgroundcolor = , %背景色と透過度 {/color[gray]{.90}}
 	breaklines = true,
 	breakindent = 10pt,
 	basicstyle = \ttfamily\scriptsize, %標準の書体
 	commentstyle = {\itshape \color[cmyk]{1,0.4,1,0}}, %コメントの書体
 	classoffset = 0, %関数名等の色の設定
 	keywordstyle = {\bfseries \color[cmyk]{0,1,0,0}}, %キーワード(int, ifなど)の書体
 	stringstyle = {\ttfamily \color[rgb]{0,0,1}}, %表示する文字の書体
 	%枠 "t"は上に線を記載, "T"は上に二重線を記載
	%他オプション:leftline,topline,bottomline,lines,single,shadowbox
 	frame = tbrl,
 	framesep = 5pt, %frameまでの間隔(行番号とプログラムの間)
 	numbers = left, %行番号の位置
 	stepnumber = 1, %行番号の間隔
 	numberstyle = \tiny, %行番号の書体
 	tabsize = 4, %タブの大きさ
 	captionpos = t
}
%参考文献を消す
\renewcommand{\refname}{}

%表中の文字の左右の余白設定
\setlength{\tabcolsep}{5mm}
